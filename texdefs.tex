% Standard environments here
\usepackage{mathtools} % pmatrix*! use \begin{matrix*}[r] to center colums right


% ----  Custom Environments ----- 
% just a horizontal rule for the "details" environment
\newcommand{\optionrule}{\noindent\rule{1.0\textwidth}{0.75pt}}
% the "details" environment declaration.
\newenvironment{aside}{%
  \def\FrameCommand{\hspace{2em}}
  \MakeFramed {\advance\hsize-\width \small}\optionrule}
{\\\optionrule\endMakeFramed}
% I took out the newline command. i don't know what it did.
%{\newline\optionrule\endMakeFramed}

% ---- Text/Formatting Symbols ---
\renewcommand{\eqref}[1]{Equation (\ref{#1})}
\newcommand{\figref}[1]{Figure \ref{#1}}
%\renewcommand{\etal}{{\it et al} }

% ---- Math symbols ------------

% basic algebra, set-theoretic, and functional notation
\newcommand{\set}[1]{\{#1\}} % set notation
\newcommand{\Range}{\mathcal{R}} % range (ie, of a function)
\newcommand{\setcomp}[1]{{#1}^{\mathsf{c}}} % set complement
%\newcommand{\inv}{^{-1}} % inverse
\newcommand{\inv}[1]{\ensuremath{{#1}^{-1}}} 
\newcommand{\ind}[1]{\mathbf{1}_{\left\{ #1\right\}}} % indicator function
\newcommand{\sgn}{\operatorname{sgn}}

\newcommand{\st}{\mbox{ s.t. }} % such that or subject to

% number sets
% \newcommand{\N}{\mathbb N}
% \newcommand{\Z}{\mathbb Z}
% \newcommand{\Q}{\mathbb Q}
% \newcommand{\R}{\mathbb R}
% \newcommand{\C}{\mathbb C}

%\DeclareMathOperator*{\argmax}{argmax}
%\DeclareMathOperator*{\argmin}{argmin}
\DeclareMathOperator*{\supp}{supp}
\DeclareMathOperator*{\tr}{tr}

% Calculus: Derivatives \& Integrals
\newcommand{\dm}[1]{\ensuremath{\,\mathrm{d}{#1}}} % Pretty symbol for differentials - ie, dx
\newcommand{\deriv}[2]{\frac{d #1}{d #2}}
\newcommand{\D}[2]{\frac{\partial #1}{\partial #2}}
\newcommand{\DD}[2]{\frac{\partial ^2 #1}{\partial #2 ^2}}
\newcommand{\Di}[2]{\frac{\partial ^i #1}{\partial #2 ^i}}
\newcommand{\evalat}[1]{\left.#1\right|}
\newcommand{\parderiv}[2]{\frac{\partial #1}{\partial{#2}}}
\newcommand{\parDeriv}[3]{\frac{\partial^{#3} #1}{\partial{#2}^{#3}}}
\newcommand{\parwrt}[1]{\frac{\partial}{\partial{#1}}}
\newcommand{\parpowrt}[2]{\frac{\partial^{#1}}{\partial {#2}^{#1}}}
\newcommand{\partwowrt}[2]{\frac{\partial^{2}}{\partial {#2} \partial{#1}}}

% basic linear algebra notation
\newcommand{\ones}{\mathop{\mathbf{1}}} % vector of ones
\newcommand{\diag}{\mathop{\mbox{diag}}}
\newcommand{\rank}{\mathop{\mathrm{rank}}} % null space ( range is \Range, above)
\newcommand{\Null}{\mathcal N} % null space ( range is \Range, above)
\newcommand{\Tr}{\textrm{Tr}} % trace
\newcommand{\norm}[1]{\left|\left|#1\right|\right|}  % normroduct
\newcommand{\tp}[1]{\ensuremath{{#1}^\top}} % transpose with argument
\newcommand{\trp}{^{\mathrm{T}}} % Transpose
\newcommand{\itrp}{^{-\mathrm{T}}} % inverse-transpose
\newcommand{\inprod}[2]{\langle #1,#2\rangle}

% Probability notation
\newcommand{\pialg}{\pi\text{-algebra}}
\newcommand{\sigalg}{\sigma\text{-algebra}}
\newcommand{\alg}[1]{\mathcal{#1}} % sigma algebra notation
\newcommand{\E}{\mathbb{E}}
\newcommand{\V}{\mathbb{V}}
\newcommand{\Prob}[1]{\mathbb{P}\left[#1\right]}
\newcommand\independent{\protect\mathpalette{\protect\independenT}{\perp}}
\def\independenT#1#2{\mathrel{\rlap{$#1#2$}\mkern2mu{#1#2}}}
\newcommand{\Norm}[2]{\mathcal{N}\left(#1, #2\right)} % normal distribution
%\newcommand{\mvn}[4]{\frac{1}{(2\pi)^{\frac{#4}{2}} |#3|^\frac{1}{2}} 
%\exp \left[-\tfrac{1}{2} (#1 - #2)\trp #3^{-1}(#1-#2)\right]} % multivariate normal
%% Notation for specific projects
%\newcommand{\mvnzero}[2]{\frac{1}{|2\pi #2|^\frac{1}{2}}
%\exp \left[-\tfrac{1}{2} #1\trp #2^{-1}#1\right]}  % zero-mean mvn

%Yuanjun's stuff
\newcommand{\N}{\mathcal{N}}

% matrix variables
\newcommand{\identityMat}{\ensuremath{I}}

\usepackage{forloop}
\newcommand{\defvec}[1]{\expandafter\newcommand\csname v#1\endcsname{{\mathbf{#1}}}}
\newcounter{ct}
\forLoop{1}{26}{ct}{
    \edef\letter{\alph{ct}}
    \expandafter\defvec\letter
}
\newcommand{\defmat}[1]{\expandafter\newcommand\csname m#1\endcsname{{\mathbf{#1}}}}
\forLoop{1}{26}{ct}{
    \edef\letter{\Alph{ct}}
    \expandafter\defmat\letter
}
\newcommand{\vmu}{\bm{\mu}}
\newcommand{\veta}{\bm{\eta}}
\newcommand{\vepsilon}{\bm{\epsilon}}
\newcommand{\vlambda}{\bm{\lambda}}


\newcommand{\mLambda}{\bm{\Lambda}}

% mutual information symbols (for MI estimation)

% convex analysis symbols (for optimization class)
\newcommand{\conv}{\text{conv}} % convex hull
\newcommand{\aff}{\text{Aff }} % affine hull
\newcommand{\extp}{\text{Ext}} %extreme points
%\newcommand{\argmin}{\mathrm{arg}\min}
\newcommand{\argmin}[1]{\underset{#1}{\operatorname{argmin}}}
\newcommand{\argmax}[1]{\underset{#1}{\operatorname{argmax}}}
\newcommand{\grad}{\nabla}
\newcommand{\curl}{\nabla\times}
\newcommand{\logdet}[1]{\log |#1|}


%% My own TheoremStyle
  \newtheoremstyle{evandefinition}{\topsep}{\topsep}%
     {}%         Body font  (\itshape)
     {}%         Indent amount (empty = no indent, \parindent = para indent)
     {\bfseries}% Thm head font
     {}%        Punctuation after thm head
     {\newline}%  Space after thm head  (default: 5pt plus 1pt minus 1pt)
     {\thmname{#1}\thmnumber{ #2}. \textit{\thmnote{ #3} }}%         Thm head spec


     \newtheoremstyle{indenteddefinition}{\topsep}{\topsep}
     {\addtolength{\leftskip}{2em}} %         Body font  (\itshape)
     {-1.75em}%         Indent amount (empty = no indent, \parindent = para indent)
     {\bfseries}% Thm head font % also: \scshape
     {}%        Punctuation after thm head
     { }%  Space after thm head  (default: 5pt plus 1pt minus 1pt)
     {\thmname{#1} \thmnumber{#2}. \textbf{(\thmnote{#3})}}%         Thm head spec
     % {} %         Thm head spec


% %% This code will allow me to have a bar around shit (uses ntheorem)
% %\theoremstyle{plain}
% \theorembodyfont{}
% \usepackage{color}
% \definecolor{gray}{rgb}{0.7,0.7,0.7}
% % This code turns the box around the theorem into a bar to the left
% \renewcommand*\FrameCommand{{\color{gray}\vrule width 1pt\hspace{10pt}}}


% general headings
%\newtheorem{theorem}{Theorem}
%\newtheorem{lemma}{Lemma}
%\newtheorem{proposition}{Proposition}
%\newtheorem{corollary}{Corollary}

%\theoremstyle{evandefinition}
%\newtheorem{definition}{Definition}
%\theoremstyle{definition}
% \theoremindent1em
% \theorembodyfont{\normalfont}
%\newframedtheorem{definition}{Definition}

%\theoremstyle{indenteddefinition} 
%\newtheorem{example}{Example}

\theoremstyle{remark} 
\newtheorem{exercise}{Exercise}
\newtheorem{remark}{Remark}

%% Make it so matrix has an align-column option ([c])
\makeatletter
\renewcommand*\env@matrix[1][c]{\hskip -\arraycolsep
  \let\@ifnextchar\new@ifnextchar
  \array{*\c@MaxMatrixCols #1}}
\makeatother

%%%%%%%%%%%%%%%%%%%%%%%%%%%%%%%%
%% Examples of things I might want to do.
%%%%%%%%%%%%%%%%%%%%%%%%%%%%%%%%
% 
%%%%%%%%%%%%%%%%
%  Function with different cases:
%%%%%%%%%%%%%%%%
% \begin{align*}
%   q(\mu) &= \inf_x L(x,\mu)
% \\
% &=\left\{
% \begin{matrix}
% &c_0 + \mu c_1 - (b_0 - \mu b_1)^T ( A_0 + \mu A_1)^+ (b_0 + \mu
% b_1 )^T  &  A_0 + \mu A_1 \succeq 0,
% \\
% &\quad & b_0 + \mu b_1 \in \mathcal{R}(A_0 + \mu A_1)
% \\
% & \infty & \mathrm{otherwise}
% \end{matrix}\right.
% \end{align*}
% 
